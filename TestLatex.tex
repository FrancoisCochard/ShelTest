\documentclass[10pt,a4paper]{shelyak}
\usepackage[utf8]{inputenc}
\usepackage[frenchb]{babel}
\usepackage[output-decimal-marker={,}]{siunitx} % version française
%\usepackage{siunitx} % version anglaise
\usepackage{amsmath}
\usepackage{amsfonts}
\usepackage{amssymb}

\begin{document}
\title{\textbf{Alpy 600 spectroscope}\\
\textbf{Manuel Utilisateur }}


\author{\textbf{Olivier Thizy}\\
(olivier.thizy@shelyak.com)\\
\\
\textbf{François Cochard}\\
(francois.cochard@shelyak.com)}


\DocReference{DC0016}
\DocRevision{Z}
\maketitle



\chapter{Introduction}
\section{Pour voir}
Je commence à écrire du texte pour vérifier que c'est sur deux colonnes -- et alors, quel résultat ?
J'affiche un nombre: 3,6 km/s
J'ai tout de même à vérifier que les langues sont respectées...
\si{\nm} (nanometer : \SI{1}{\nm} =
    \SI{e-9}{\metre}). However, in your practice of spectroscopy, you
    will find both \si{\nm} and Angstr\"oms (\si{\angstrom}). One Angstr\"om
    is equal to \SI{3,75 e-10}{\m}, therefore the relationship between the
    two is easy:  \SI{1}{\nm} = \SI{10}{\angstrom}. Regularly, purist
    fight over this, but actually this duality is not really a problem
    -- you get used to it quickly. In this book, I chose to use
    \si{\nm} -- but in some figures the wavelength is given in \si{\angstrom}.
    
    \footnote{toto}
\end{document}